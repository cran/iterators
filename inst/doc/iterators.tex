% \VignetteIndexEntry{iterators Manual}
% \VignetteDepends{iterators}
% \VignettePackage{iterators}
\documentclass[12pt]{article}
\usepackage{amsmath}
\usepackage[pdftex]{graphicx}
\usepackage{color}
\usepackage{xspace}
\usepackage{fancyvrb}
\usepackage{fancyhdr}
    \usepackage[
         colorlinks=true,
         linkcolor=blue,
         citecolor=blue,
         urlcolor=blue]
         {hyperref}
         \usepackage{lscape}

\usepackage{Sweave}

%%%%%%%%%%%%%%%%%%%%%%%%%%%%%%%%%%%%%%%%%%%%%%%%%%%%%%%%%%%%%%%%%%

% define new colors for use
\definecolor{darkgreen}{rgb}{0,0.6,0}
\definecolor{darkred}{rgb}{0.6,0.0,0}
\definecolor{lightbrown}{rgb}{1,0.9,0.8}
\definecolor{brown}{rgb}{0.6,0.3,0.3}
\definecolor{darkblue}{rgb}{0,0,0.8}
\definecolor{darkmagenta}{rgb}{0.5,0,0.5}

%%%%%%%%%%%%%%%%%%%%%%%%%%%%%%%%%%%%%%%%%%%%%%%%%%%%%%%%%%%%%%%%%%

\newcommand{\bld}[1]{\mbox{\boldmath $#1$}}
\newcommand{\shell}[1]{\mbox{$#1$}}
\renewcommand{\vec}[1]{\mbox{\bf {#1}}}

\newcommand{\ReallySmallSpacing}{\renewcommand{\baselinestretch}{.6}\Large\normalsize}
\newcommand{\SmallSpacing}{\renewcommand{\baselinestretch}{1.1}\Large\normalsize}

\newcommand{\halfs}{\frac{1}{2}}

\setlength{\oddsidemargin}{-.25 truein}
\setlength{\evensidemargin}{0truein}
\setlength{\topmargin}{-0.2truein}
\setlength{\textwidth}{7 truein}
\setlength{\textheight}{8.5 truein}
\setlength{\parindent}{0.20truein}
\setlength{\parskip}{0.10truein}

%%%%%%%%%%%%%%%%%%%%%%%%%%%%%%%%%%%%%%%%%%%%%%%%%%%%%%%%%%%%%%%%%%
\pagestyle{fancy}
\lhead{}
\chead{Using The {\tt iterators} Package}
\rhead{}
\lfoot{}
\cfoot{}
\rfoot{\thepage}
\renewcommand{\headrulewidth}{1pt}
\renewcommand{\footrulewidth}{1pt}
%%%%%%%%%%%%%%%%%%%%%%%%%%%%%%%%%%%%%%%%%%%%%%%%%%%%%%%%%%%%%%%%%%

\title{Using The {\tt iterators} Package}
\author{Rich Calaway \\ richcalaway@revolution-computing.com}


\begin{document}

\maketitle

\thispagestyle{empty}
\section{Introduction}

An {\em iterator} is a special type of object that generalizes the notion of 
a looping variable. When passed as an argument to a function that knows what 
to do with it, the iterator supplies a sequence of values. The iterator also 
maintains information about its state, in particular its current index. The
\texttt{iterators} package includes a number of functions for creating 
iterators, the simplest of which is \texttt{iter}, which takes
virtually any R object and turns it into an iterator object. The simplest 
function that operates on iterators is the \texttt{nextElem} function, which 
when given an iterator, returns the next value of the iterator. For example, 
here we create an iterator object from the sequence 1 to 10, and then use 
\texttt{nextElem} to iterate through the values:
\begin{Schunk}
\begin{Sinput}
> library(iterators)
> i1 <- iter(1:10)
> nextElem(i1)
\end{Sinput}
\begin{Soutput}
[1] 1
\end{Soutput}
\begin{Sinput}
> nextElem(i1)
\end{Sinput}
\begin{Soutput}
[1] 2
\end{Soutput}
\end{Schunk}

You can create iterators from matrices and data frames, using the \texttt{by} argument to specify whether to iterate by row or column:
\begin{Schunk}
\begin{Sinput}
> istate <- iter(state.x77, by = "row")
> nextElem(istate)
\end{Sinput}
\begin{Soutput}
        Population Income Illiteracy Life Exp Murder HS Grad Frost  Area
Alabama       3615   3624        2.1    69.05   15.1    41.3    20 50708
\end{Soutput}
\begin{Sinput}
> nextElem(istate)
\end{Sinput}
\begin{Soutput}
       Population Income Illiteracy Life Exp Murder HS Grad Frost   Area
Alaska        365   6315        1.5    69.31   11.3    66.7   152 566432
\end{Soutput}
\end{Schunk}

Iterators can also be created from functions, in which case the iterator can be an endless source of values:
\begin{Schunk}
\begin{Sinput}
> ifun <- iter(function() sample(0:9, 4, replace = TRUE))
> nextElem(ifun)
\end{Sinput}
\begin{Soutput}
[1] 1 6 5 6
\end{Soutput}
\begin{Sinput}
> nextElem(ifun)
\end{Sinput}
\begin{Soutput}
[1] 0 4 3 8
\end{Soutput}
\end{Schunk}

For practical applications, iterators can be paired with \texttt{foreach} to obtain parallel results quite easily:
\begin{Schunk}
\begin{Sinput}
> library(foreach)
\end{Sinput}
\begin{Soutput}
foreach: simple, scalable parallel programming from REvolution Computing
Use REvolution R for scalability, fault tolerance and more.
http://www.revolution-computing.com
\end{Soutput}
\begin{Sinput}
> x <- matrix(rnorm(1e+06), ncol = 10000)
> itx <- iter(x, by = "row")
> foreach(i = itx, .combine = c) %dopar% mean(i)
\end{Sinput}
\begin{Soutput}
  [1] -0.0040710604  0.0097467171  0.0022043596  0.0109891194 -0.0062254874
  [6]  0.0130028917 -0.0024575821  0.0052974462 -0.0029302729 -0.0107083931
 [11] -0.0164476128 -0.0065090243  0.0016966248 -0.0123011644 -0.0006232640
 [16]  0.0081913672  0.0030860009  0.0130874302 -0.0096216450  0.0088969712
 [21] -0.0020604662 -0.0043140207 -0.0012823852  0.0060662727 -0.0101623872
 [26] -0.0037512240  0.0016970882  0.0017170333 -0.0141240177 -0.0042973247
 [31]  0.0010162580  0.0116825920  0.0045971389 -0.0040916539  0.0053578457
 [36]  0.0034443511  0.0164102293  0.0031012621  0.0037105796  0.0096859089
 [41]  0.0120647321 -0.0008609619  0.0053445192  0.0097641619  0.0113396312
 [46]  0.0078107070  0.0139583829  0.0024135899 -0.0208531016 -0.0062693833
 [51]  0.0151801025 -0.0027044067  0.0059327559  0.0025392198  0.0073770923
 [56] -0.0005665411  0.0248923225 -0.0154207459  0.0127998011 -0.0090775257
 [61] -0.0077243702 -0.0175771448  0.0107690258  0.0036175829  0.0004195700
 [66] -0.0020560297  0.0070443072  0.0098662270 -0.0002170741 -0.0170780578
 [71]  0.0034730946 -0.0017520503 -0.0100526727 -0.0162407305 -0.0012446651
 [76] -0.0048357119  0.0234021283  0.0141282018 -0.0056361458  0.0071032796
 [81] -0.0089421043  0.0059342475  0.0070410147 -0.0070316701  0.0065529424
 [86]  0.0288267901 -0.0097902841 -0.0058481502  0.0064289241 -0.0103422987
 [91]  0.0008049182 -0.0045195646 -0.0130103143  0.0074859619  0.0086388529
 [96]  0.0058278355  0.0125116002 -0.0013952979  0.0162009874  0.0085128135
\end{Soutput}
\end{Schunk}

\section{Some Special Iterators}

The notion of an iterator is new to R, but should be familiar to users of
languages such as Python. The \texttt{iterators} package includes a number of
special functions that generate iterators for some common scenarios. For 
example, the
\texttt{irnorm} function creates an iterator for which each value is drawn
from a specified random normal distribution:
\begin{Schunk}
\begin{Sinput}
> library(iterators)
> itrn <- irnorm(10)
> nextElem(itrn)
\end{Sinput}
\begin{Soutput}
 [1]  1.0350858 -0.2464664 -2.2801741 -1.5195656 -1.1507377  0.7881754
 [7] -0.8338092 -0.8315724  0.1029678  0.7088185
\end{Soutput}
\begin{Sinput}
> nextElem(itrn)
\end{Sinput}
\begin{Soutput}
 [1] -0.8444356 -1.3008714 -0.2281682 -0.1154227  0.3875200  0.8456657
 [7]  0.6657748  1.1860130 -2.0066959  0.3022127
\end{Soutput}
\end{Schunk}

Similarly, the \texttt{irunif}, \texttt{irbinom}, and \texttt{irpois} functions
create iterators which drawn their values from uniform, binomial, and Poisson
distributions, respectively.

We can then use these functions just as we used \texttt{irnorm}:
\begin{Schunk}
\begin{Sinput}
> itru <- irunif(10)
> nextElem(itru)
\end{Sinput}
\begin{Soutput}
 [1] 0.12272881 0.23282231 0.88774034 0.41627943 0.80330098 0.55480407
 [7] 0.93245959 0.08326419 0.03605037 0.55380334
\end{Soutput}
\begin{Sinput}
> nextElem(itru)
\end{Sinput}
\begin{Soutput}
 [1] 0.31878022 0.06503089 0.11131593 0.68338828 0.87976008 0.54240576
 [7] 0.25652083 0.49187155 0.87888189 0.34929741
\end{Soutput}
\end{Schunk}

The \texttt{icount} function returns an iterator that counts starting from one:
\begin{Schunk}
\begin{Sinput}
> it <- icount(3)
> nextElem(it)
\end{Sinput}
\begin{Soutput}
[1] 1
\end{Soutput}
\begin{Sinput}
> nextElem(it)
\end{Sinput}
\begin{Soutput}
[1] 2
\end{Soutput}
\begin{Sinput}
> nextElem(it)
\end{Sinput}
\begin{Soutput}
[1] 3
\end{Soutput}
\end{Schunk}

\end{document}
